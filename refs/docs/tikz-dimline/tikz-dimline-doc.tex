
\documentclass[a4paper]{article}

\usepackage[all]{nowidow}
\widowpenalty10000
\clubpenalty10000


%% See http://tex.stackexchange.com/questions/78334/print-system-information-in-latex-doc
\RequirePackage{ifxetex}
\ifxetex
\RequirePackage{fontspec}
\RequirePackage{xltxtra}
\newcommand{\myxetex}{\XeLaTeX}
\newcommand{\compilator}{\XeLaTeX{}}
\newcommand{\cversion}{\the\XeTeXversion\XeTeXrevision}

\else
\RequirePackage[utf8]{inputenc}
\newcommand{\myxetex}{\texttt{XeLaTeX}}
\newcommand{\compilator}{\LaTeX{}}
\makeatletter
\newcommand{\cversion}{
\ifnum\pdftexversion<100 %
          \the\pdftexversion.\pdftexrevision
        \else
          \ifnum\pdftexversion<130 %
            \expandafter\@car\the\pdftexversion\@empty\@nil.%
            \expandafter\@cdr\the\pdftexversion\@empty\@nil  
            \pdftexrevision
          \else
            \expandafter\@car\the\pdftexversion\@empty\@nil.%
            \expandafter\@cdr\the\pdftexversion\@empty\@nil.%
            \pdftexrevision
          \fi
        \fi
}
\makeatother
\fi

\makeatletter
\newcommand{\tcbversion}{\tcb@version}
\makeatother

\RequirePackage{etoolbox}

\RequirePackage{minted}


\RequirePackage{tikz}
\usetikzlibrary{calc}
\usetikzlibrary{fit,spy}
\usetikzlibrary{intersections}

\usepackage{pgfplots}

\RequirePackage{tcolorbox}
\tcbuselibrary{skins,listings}
\tcbuselibrary{minted}
\tcbuselibrary{documentation}

%% Diameter
\usepackage{ wasysym }

\usepackage{tikz-dimline}


%%\tikzset{every picture/.append style={scale=0.5}}
%% \tikzset{global scale/.style={
%%     scale=0.5,
%%     every node/.style={scale=0.5}
%%   }
%% }

%\tcbset{listing engine=latex}
\newminted[srcGrid]{latex}{baselinestretch=0.5,fontsize=\footnotesize}


\newtcblisting{exampleB}[2][]{%
  colframe=red!50!yellow!50!black,
  colback=white,
  coltitle=red!50!yellow!3!white,
  bicolor,colbacklower=red!50!yellow!5!white,
  fonttitle=\sffamily\bfseries,
  center upper,
  minted language=latex,
  minted options={baselinestretch=0.5,fontsize=\scriptsize},
  %%lefthand ratio=0.33,
  %%sidebyside,
  text and listing,
  title=#2,#1}

\newtcblisting{exampleA}[2][]{%
  colframe=red!50!yellow!50!black,
  colback=white,
  coltitle=red!50!yellow!3!white,
  bicolor,colbacklower=red!50!yellow!5!white,
  fonttitle=\sffamily\bfseries,
  center upper,
  minted language=latex,
  minted options={baselinestretch=0.5,fontsize=\scriptsize},
  %%lefthand ratio=0.33,
  %%sidebyside,
  listing,
  title=#2,#1}


%% \newtcblisting{exampleA}[2][]{%
%%   colframe=red!50!yellow!50!black,
%%   colback=white,
%%   coltitle=red!50!yellow!3!white,
%%   bicolor,colbacklower=red!50!yellow!5!white,
%%   fonttitle=\sffamily\bfseries,
%%   sidebyside,text and listing,
%%   title=#2,#1}





\BeforeBeginEnvironment{srcGrid}{%
\begin{tcolorbox}[boxrule=0.5pt,left=1mm,width=\linewidth]}
\AfterEndEnvironment{srcGrid}{\end{tcolorbox}}%





%% \AtBeginEnvironment{exampleB}{}
%% \AtEndEnvironment{exampleB}{}


%% \BeforeBeginEnvironment{srcGrid}{%
%% \begin{exampleB}[boxrule=0.5pt,left=1mm,width=\linewidth]{}}
%% \AfterEndEnvironment{srcGrid}{\end{exampleB}}%



%% Just to draw a help grid
%% \helpgrid{Xa}{Ya}{Xb}{Yb}
\makeatletter
\newcommand{\helpgrid@do@nothing}[1]{#1}%
\newcommand{\helpgrid}[4]{%
  \pgfplotsifinaxis{%
    \let\helpgrid@internal\pgfplotsextra%
  }{%
    \let\helpgrid@internal\helpgrid@do@nothing%
  }%

  \begin{scope}

    \coordinate (a) at (#1,#2) {};
    \coordinate (b) at (#3,#4) {};

    \draw[step=1,gray!10,very thin] (a) grid (b);
    
    \foreach \x in {#1,...,#3}
    \node[anchor=north,gray!20] at (\x, 0) {\tiny{\x}};

    \foreach \y in {#2,...,#4}
    \node[anchor=east,gray!20] at (0, \y) {\texttt{\tiny{$\y$}}};
  \end{scope}
}%
\makeatother


\newcommand{\PGF}{\textsc{PGF}}
\newcommand{\TikZ}{Ti\textcolor{orange}{\emph{k}}Z}



\title{
  \texttt{dimline}\\
  Technical dimension lines using \PGF/\TikZ\\
}
\author{Sébastien Gross\\ \texttt{<seb•$\alpha\tau$•chezwam•$\partial\theta\tau$•org>}}
\date{\csname ver@tikz-dimline.sty\endcsname}

\begin{document}

\maketitle


\let\thefootnote\relax\footnotetext{This documentation was compiled on
  {\platformname} using {\compilator} {\cversion}, {\PGF} {\pgfversion},
  \texttt{tcolorbox} {\tcbversion} on {\today}.}


Dimension lines are drawn segments that indicate the measurement of a
feature. The line has an arrow at both ends to show that the dimension
displayed covers the entire span of that line. The direction of the line
(vertical, horizontal or diagonal) shows the direction of the
measurement. Dimension lines are kept outside of the object illustrated,
typically connected to other lines that display the area where that
measurement applies. This keeps the illustration clear.

\begin{tcolorbox}[title={Contents},fonttitle=\bfseries\Large,
  colback=yellow!10!white,colframe=red!50!black,before=\par\bigskip\noindent]
  \makeatletter
  \@starttoc{toc}
  \makeatother
  %\tableofcontents
\end{tcolorbox}


\newpage



\section{Overview}

\begin{exampleB}{}

\begin{tikzpicture}[scale=2]
  \draw[line width=1pt]  (0,0) -- (0,4) -- (.5,4) -- (.5,5) -- (1.,5)
  -- (1.,4) -- (1.5,4) -- (1.5,0) -- cycle;

  \draw node (A) at (2.5,0) {};
  
  \dimline[line style = {line width=0.7},
    extension start length=-0.24,
    extension end length=-0.24] {(A)}{($ (A) + (0,4) $)}{$4.0$};

  \dimline[line style = {red, line width=0.7},
    extension start length=0,
    extension end length=0] {(0,2)}{(1.5, 2)}{ $\diameter 1.5$};

  \dimline[label style={above=0.5ex,},
    extension start path = {(0,6) (0,5.7) (0.5,5.4) (0.5,5.05)},
    extension end path =  {(1.5,6) (1.5,5.7) (1,5.4) (1,5.05)},]
          {(0,6)}{(1.5,6)}{$\diameter 0.5$};
\end{tikzpicture}
\end{exampleB}


\let\thefootnote\relax\footnotetext{The \texttt{wasysym} package is required to have \texttt{\textbackslash diameter} command.}


\newpage


\section{Installation}

To install the \verb"tkiz-dimline" package copy its directory to either to
\begin{itemize}
\item \verb"$TEXHOME/tex/latex/"
\item \verb"$TEXMFHOME/tex/latex/"
\item \verb"~/texmf/tex/latex/"
\item \verb"~/Library/texmf/tex/latex/"
\end{itemize}

\section{Usage}

\subsection{command}
\begin{docCommand}{dimline}{\oarg{key path}\marg{start}\marg{end}\marg{label}}
  Creates a new dimension line from \meta{start} to \meta{end} with
  \meta{label}. An optional \meta{key path} can be used to customise the
  dimension line.

  This command has to be called within a \docAuxEnvironment*{tikzpicture}
  environment.

\end{docCommand}

\subsection{options}
\begin{docKey}{color}{=\meta{color}}{black}
  Dimension line color.
\end{docKey}
\begin{docKey}{line style}{=\meta{style}}{not set}
  Dimension line style, identical to {\PGF} line style.
\end{docKey}
\begin{docKey}{label style}{=\meta{style}}{not set}
  Dimension label style, identical to {\PGF} label style.
\end{docKey}

\begin{docKey}{extension start length}{=\meta{size}}{1cm}
  Length of the extention line at the start point.
\end{docKey}

\begin{docKey}{extension end length}{=\meta{size}}{1cm}
  Identical to \refKey*{extension start length}, but for the end point.
\end{docKey}

\begin{docKey}{extension start angle}{=\meta{angle}}{-90}
  Angle between the dimension line and the extention line at the start
  point.  You dont't want to change that value unless you really know what
  you are doing. You would rather have a look to \refKey*{extension start
    path} and \refKey*{extension end path}.
\end{docKey}

\begin{docKey}{extension end angle}{=\meta{angle}}{90}
  Identical to \refKey*{extension start angle}, but for the end point.
\end{docKey}

\begin{docKey}{extension start style}{=\meta{style}}{not set}
  Style for the extension line at the start point.
\end{docKey}

\begin{docKey}{extension end style}{=\meta{style}}{not set}
  Identical to \refKey*{extension start style}, but for the end point.
\end{docKey}

\begin{docKey}{extension start path}{=\meta{path}}{not set}
  Path (coordinates relative to the current
  \docAuxEnvironment*{tkizpicture}) for the extension line at the start
  point.
\end{docKey}

\begin{docKey}{extension end path}{=\meta{path}}{not set}
  Identical to \refKey*{extension start path}, but for the end point.
\end{docKey}


\begin{docKey}{arrows}{=\meta{arrows spec}}{dimline-dimline}
  Arrows used for the dimension line. \docAuxKey{dimline-dimline} is used
  for standard arrows. Use \docAuxKey{dimeline reverse-dimline reverse} to
  reverse the arrow heads.
\end{docKey}



\section{Examples}

On following examples \docAuxCommand{helpgrid} is only used for a labeled
grid. This not mandatory you can dicard this line. See source of this
documentation to see how \docAuxCommand{helpgrid} command is defined.



\subsection{Basic usage}

This is a very simple usage. It just draws a dimension line (using default
values) between 2 points with a label on it. Note that the dimension starts
at the border of the side tick mark, not at the end of the arrow.

\begin{exampleB}{}
  \begin{tikzpicture}[spy using outlines={circle, magnification=25, size=6cm,
        connect spies}]
    \helpgrid{-1}{-1}{5}{4}
    \draw[line width=0.01,color=red] plot[mark=+] coordinates{(3,3)} {};
    \draw[line width=0.01,color=green] plot[mark=x] coordinates{(3,3)} {};
    \dimline[line style = {line width=0.5}]{(0,0)}{(3,3)}{Label};
    \spy [blue] on (2.975,2.975) in node [fill=white] at (6.5,1.5);
  \end{tikzpicture}
\end{exampleB}


\subsection{Reverse arrows}

Sometimes, shuch as for very small dimensions, you need to reverse the
arrows. You can then use a \docAuxKey{dimeline reverse} arrow.

\begin{exampleB}{}
  \begin{tikzpicture}[spy using outlines={circle, magnification=25, size=6cm,
        connect spies}]
    \helpgrid{-1}{-1}{5}{4}
    \draw[line width=0.01,color=red] plot[mark=+] coordinates{(3,3)} {};
    \draw[line width=0.01,color=green] plot[mark=x] coordinates{(3,3)} {};

    \dimline[line style = {line width=0.5,
    arrows=dimline reverse-dimline reverse}]{(0,0)}{(3,3)}{Label};
    \spy [blue] on (3.025,3.025) in node [fill=white] at (6.5,1.5);
  \end{tikzpicture}
\end{exampleB}

\subsection{Styling your dimension lines} 

You can customize your dimension lines like any other \TikZ{} environment.
\begin{exampleB}{}
  \begin{tikzpicture}[scale=0.5]
    \helpgrid{-1}{-1}{4}{7}
    \draw (0,0) -- (0,4) -- (.5,4) -- (.5,5) -- (1.,5) -- (1.,4) -- (1.5,4) -- (1.5,0)
    -- cycle;
    \dimline[color=blue,
      extension start length=-0.25, extension end length=-0.25]{(2.75,0)}{(2.75,4)}{4.0};
    \dimline[color=red,
      line style={arrows=dimline reverse-dimline reverse},
      label style={above=0.8ex,font=\tiny},
      extension start length=1,
      extension end length=1]{(0.5,5.6)}{(1.0,5.6)}{0.5};
  \end{tikzpicture}
\end{exampleB}


\subsection{Complex extension lines}

This example shows you how to use both a \docAuxKey{extension start path}
and \docAuxKey{extension end path}. This is useful if you want to enlarge a
small dimension area.

\begin{exampleB}{}
  \begin{tikzpicture}[scale=0.5]
    \helpgrid{-1}{-1}{4}{7}
    \draw  (0,0) -- (0,4) -- (.5,4) -- (.5,5) -- (1.,5) -- (1.,4) -- (1.5,4) -- (1.5,0)
    -- cycle;

    \draw[color=red] node (A) at (2.5,0) {} node [yshift=1ex,xshift=-1ex]
    at (A) {\tiny{A}};
    \dimline[color=blue,
      line style = {line width=0.7},
      extension start length=-0.24,
      extension end length=-0.24] {(A)}{($ (A) + (0,4) $)}{4.0};
    \dimline[label style={above=0.5ex, fill=blue!10,},
      extension start path = {(0,6) (0,5.7) (0.5,5.4) (0.5,5.05)},
      extension start style = { green },
      extension end path =  {(1.5,6) (1.5,5.7) (1,5.4) (1,5.05)},]
            {(0,6)}{(1.5,6)}{0.5};
    \draw[color=red,line width=0.01] plot[mark=x] coordinates{(A)} {};
  \end{tikzpicture}
\end{exampleB}

\section{License}

This program is free software. It comes without any warranty, to the extent
permitted by applicable law. You can redistribute it and/or modify it under
the terms of the Do What The Fuck You Want To Public License, Version 2, as
published by Sam Hocevar. See http://sam.zoy.org/wtfpl/COPYING for more
details.



\end{document}
