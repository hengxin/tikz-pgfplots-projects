%% example2_circular.tex
%% Copyright 2015 J. Schmidt
%% tikz-palattice LaTeX package
% 
% This work may be distributed and/or modified under the
% conditions of the LaTeX Project Public License, either version 1.3
% of this license or (at your option) any later version.
% The latest version of this license is in
% http://www.latex-project.org/lppl.txt
% and version 1.3 or later is part of all distributions of LaTeX
% version 2005/12/01 or later.
% 
% This work has the LPPL maintenance status `maintained'.
% 
% The Current Maintainer of this work is J. Schmidt.
% 
% This work consists of the files tikz-palattice.sty and tikz-palattice_documentation.tex
% and the 5 example files example1_linear.tex, example2_circular.tex,
% example3_coordinates.tex, example4_labels.tex and elsa.tex

\documentclass[]{standalone}
\usepackage[ngerman]{babel}
\usepackage[utf8]{inputenc}
\usepackage{tikz-palattice}


\begin{document}

\begin{lattice}
  \setlinecolor{drift}{red}

  \drift{1}
  \dipole{M1}{2}{45}
  \drift{1}
  \dipole{M2}{2}{45}
  \drift{1}
  \setelementcolor{dipole}{green!50!black} % change color of all following dipoles
  \dipole{M3}{2}{45}
  \drift{1}
  \dipole{M4}{2}{45}
  \drift{1}
  \begin{scope}                        % color change only applied within scope
    \setelementcolor{dipole}{red}[red] % optional 3rd argument sets top color -> no (visible) color gradient
    \dipole{M5}{2}{45}
  \end{scope}
  \drift{1}
  \dipole{M6}{2}{45}
  \drift{1}
  \resetelementcolor{dipole} % reset color of following dipoles to default
  \dipole{M7}{2}{45}[r]      % rectangle dipole type
  \drift{1}
  \dipole{M8}{2}{45}[br]     % bend rectangle dipole type

  \drawrule{(1,-0.5)}
\end{lattice}

\end{document}



