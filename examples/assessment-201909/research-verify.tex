% research-verify.tex

\documentclass[tikz]{standalone}
\usepackage{xeCJK}
\usetikzlibrary{positioning, arrows.meta, fit}

\def\y{3.5}
\def\yweak{1.2}
\def\ystrong{2.5}

\def\x{5.5}
\def\xprinciple{1.5}
\def\xvariant{3.5}

\begin{document}
\begin{tikzpicture}[>=Stealth, node distance = 0.70cm and 0.00cm,
  point/.style = {circle, scale = 0.50, fill = #1}]
  % axis
  \draw [<->,thick] (0, \y) node (yaxis) [above] {}
	  |- (\x, 0) node (xaxis) [right] {};

  % y axis
  \node (weak) [point = blue] at (0, \yweak) {};
  \node (weaklabel) [align = center, left = 5pt of weak] {{\it 弱一致性} \\ ({\it\footnotesize 最终一致性})};
  \node (strong) [point = blue] at (0, \ystrong) {};
  \node (weaklabel) [align = center, left = 5pt of strong] {{\it 强一致性} \\ ({\it\footnotesize 分布式共识})};

  % spec
  \node (framework) [draw, thick, rectangle, dashed, blue, fit = (weak) (strong), inner sep = 5pt,
    label = {[yshift = -3pt, blue] below : {规约框架}}] {};

  % x axis
  \node (principle) [point = teal, label = {[teal] below : {\it 原理}}] at (\xprinciple, 0) {};
  \node (variants) [point = red, scale = 1.5, label = {[] below : {\it 变体}}] at (\xvariant, 0) {};

  \node (lweak) [point = teal] at (\xprinciple, \yweak) {};
  \node (lstrong) [point = teal] at (\xprinciple, \ystrong) {};
  \node (l) [draw, rectangle, fit = (lweak) (lstrong), ] {};

  \node (rweak) [point = red, scale = 1.5] at (\xvariant, \yweak) {};
  \node (rstrong) [point = red, scale = 1.5] at (\xvariant, \ystrong) {};
  \node (r) [draw, rectangle, fit = (rweak) (rstrong), ] {};

  % derive 
  \draw [double, -{Implies[]}, double distance = 5pt, shorten >= 2pt, shorten <= 4pt] 
	(l) to node [above = 2pt] {\small 真实场景} node [below = 2pt] {\small 派生} (r);
\end{tikzpicture}
\end{document}