% long-fork.tex

\documentclass{standalone}
% newcommands.tex

\newcommand{\enq}{\texttt{enq}}
\newcommand{\deq}{\texttt{deq}}
\newcommand{\pput}{\texttt{PUT}}
\newcommand{\get}{\texttt{GET}}
\newcommand{\vs}{\texttt{vis}}
\newcommand{\so}{\texttt{so}}
\newcommand{\arb}{\texttt{ar}}
\newcommand{\rf}{\texttt{rf}}

% example
\newcommand{\po}[2]{\draw [->, thick] (#1) to node[above] {\Large{\so}} (#2);}
\newcommand{\pva}[2]{\draw [->, thick] (#1) to node[above] {$\Large{\so},\Large{\vs},\Large{\arb}$} (#2);}
\newcommand{\pbva}[2]{\draw [->, thick] (#1) to node[above] {$\Large{\so}$} node[below] {$\Large{\vs},\Large{\arb}$} (#2);}
\newcommand{\pv}[2]{\draw [->, thick] (#1) to node[above] {\Large{\so}} node[below] {\Large{\vs}} (#2);}
\newcommand{\evis}[2]{\draw [->, thick] (#1) to node[above, sloped, near end] {\Large{\vs}} (#2);}
\newcommand{\mvis}[2]{\draw [->, thick] (#1) to node[above, sloped] {\Large{\vs}} (#2);}
\newcommand{\ar}[2]{\draw [->, thick, allow upside down] (#1) to node[above, sloped] {\Large{\arb}} (#2);}
\newcommand{\va}[2]{\draw [->, thick, allow upside down] (#1) to node[above, sloped] {$\Large{\vs},\Large{\arb}$} (#2);}
\newcommand{\vab}[2]{\draw [->, thick, allow upside down] (#1) to node[below, sloped, near end] {$\Large{\vs},\Large{\arb}$} (#2);}
\newcommand{\vae}[2]{\draw [->, thick, allow upside down] (#1) to node[above, sloped, near end] {$\Large{\vs},\Large{\arb}$} (#2);}
\newcommand{\vas}[2]{\draw [->, thick, allow upside down] (#1) to node[sloped, near start, above] {$\Large{\vs},\Large{\arb}$} (#2);}

% serialization
\newcommand{\scc}[2]{\draw [->, very thick] (#1) to (#2);}
\newcommand{\rva}[2]{\draw [->, thick, allow upside down] (#1) to node[above, sloped] {$\Large{\rf},\Large{\vs},\Large{\arb}$} (#2);}
\newcommand{\rvb}[2]{\draw [->, thick, allow upside down] (#1) to node[below, sloped] {$\Large{\rf},\Large{\vs},\Large{\arb}$} (#2);}


\usepackage{tikz}
\usetikzlibrary{shapes, positioning, arrows.meta, decorations.pathmorphing}

\begin{document}
\begin{tikzpicture}[
  node distance = 0.8cm and 1.5cm,
  so/.style = {->, thick},
  wr/.style = {->, thick},
  ww/.style = {->, thick, dashed, red},
  rw/.style = {->, thick, dotted, blue},
  txn/.style = {draw, inner sep = 2pt, align = center}]

  \node (t0) at (0,0) {};

  \node[txn, label = above : $T_{1}$, above right = 0.50cm and -1.00cm of t0] (t1)
    {$\readevent(\keyxvar, 0)$ \po $\writeevent(\keyxvar, 1)$};
  \node[txn, label = below : $T_{2}$, below right = 0.50cm and -1.00cm of t0] (t2)
    {$\readevent(\keyyvar, 0)$ \po $\writeevent(\keyyvar, 1)$};

  \node[txn, right = of t1, label = above : $T_{3}$] (t3)
    {$\readevent(\keyxvar, 1)$ \po $\readevent(\keyyvar, 0)$};
  \node[txn, right = of t2, label = below : $T_{4}$] (t4)
    {$\readevent(\keyxvar, 0)$ \po $\readevent(\keyyvar, 1)$};

  % t1 WR(x) t3
  \draw[wr] (t1) to node[above]{$\WR(\keyxvar)$} (t3);
  % t2 WR(y) t4
  \draw[wr] (t2) to node[below]{$\WR(\keyyvar)$} (t4);

  % t4 RW(x) t1
  \draw[rw] (t4) to node[near start, above, sloped] {$\RW(\keyxvar)$} (t1);

  % t3 RW(y) t2
  \draw[rw] (t3) to node[near end, above, sloped] {$\RW(\keyyvar)$} (t2);
\end{tikzpicture}
\end{document}