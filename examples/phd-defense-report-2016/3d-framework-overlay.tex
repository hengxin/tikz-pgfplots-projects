% \documentclass[beamer]{standalone}
\documentclass[beamer]{standalone}

\usepackage{xeCJK}
\usepackage{zhnumber}	% counters in Chinese

\usetheme{CambridgeUS} % try Madrid
\usecolortheme{beaver} % try beaver, dolphin, seahorse
% \usefonttheme{serif}

\usepackage{tikz}
\usepackage{tikz-3dplot}
\usepackage{pgfplots}
\usetikzlibrary{arrows.meta, shapes, positioning, calc, backgrounds, fit}

\newcommand{\papertitle}{分布数据一致性技术研究}
%%%%%%%%%%%%%%%%%%%%%%%%%%%%%%%%%%%%%%%%%%%%%%%%%%%%%%%%%%%%%%%%%%%%%%%%%%%%%%%%	
\title[\papertitle]{\papertitle}
\subtitle{(博士学位论文答辩报告)}

\author[魏恒峰]{答辩人: 魏恒峰\\导师: 吕建教授、黄宇副教授}
\institute{南京大学软件所}
\date{\zhtoday}
%%%%%%%%%%%%%%%%%%%%
\begin{document}

\section{技术框架}
\begin{frame}{技术框架}
\tdplotsetmaincoords{60}{135}

\begin{figure}[h]
  \centering
{
\fontfamily{lmr}\selectfont
\begin{tikzpicture}[scale = 1.3, framed, background rectangle/.style = {draw = teal}, scale=2, tdplot_main_coords, font = \scriptsize]

\coordinate (O) at (0,0,0);
\coordinate (x) at (1,0,0);
\coordinate (y) at (0,1,0);
\coordinate (z) at (0,0,1);

\uncover<2->{
\draw[thick, >=Stealth, ->] (O) to node[sloped, below, near end, align = center] {Mechanism\\{\tiny (机制: 怎么做)}} (x);
\draw[thick, >=Stealth, ->] (O) to node[anchor = north, sloped, below, near end, align = center] {Measurement\\{\tiny (度量: 怎么样)}} (y);
\draw[thick, >=Stealth, ->] (O) to (z) node[anchor = south, align = center] {Semantics\\{\tiny (模型: 是什么)}};
}

\coordinate (xz) at (1,0,1);
\coordinate (yz) at (0,1,1);

\uncover<3->{
\draw[canvas is xz plane at y = 0, transform shape, draw = blue, fill = blue!50, opacity = 0.5] (xz) rectangle (O);
\node[canvas is xz plane at y = 0, align = center] at (0.5,0,0.5) {\reflectbox{多样化,} \\\reflectbox{可调节}};
}

\uncover<4->{
\draw[canvas is yz plane at x = 0, transform shape, draw = brown, fill = brown!50, opacity = 0.5] (yz) rectangle (O);
\node[canvas is yz plane at x = 0, align = center] at (0,0.5,0.5) {精细化,\\可度量};
}

\node[font = \normalsize, rotate = -90, teal] at (0,1.4,1.4) {\textsc{Datatype}};
\end{tikzpicture}
}
\caption{``一个基础, 三个维度''技术框架.}
\end{figure}
\end{frame}
\end{document}